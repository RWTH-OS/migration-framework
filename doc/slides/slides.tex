%%
%% if lualatex is not used, compile to PDF 1.4 (compatibility to Adobe Acrobat)
%%
\RequirePackage{ifluatex}
\ifluatex
\else
	\RequirePackage{pdf14}
\fi

\documentclass[10pt, aspectratio=1610]{beamer}
%%
%% E.ON ERC Beamer Theme (layout based on E.ON ERC PowerPoint template)
%% by Simon Pickartz, Institute for Automation of Complex Power Systems, 2014
%%   spickartz@eonerc.rwth-aachen.de 
%% based on RWTH Latex template 
%% by Georg Wassen, Lehrstuhl für Betriebssysteme, 2013
%%    wassen@lfbs.rwth-aachen.de
%%
%% The templates are derived from the beamer documentation and the provided 
%% templates, hence, the same licence applies:
%%
%% ----------------------------------------------------------------
%% |  This file may be distributed and/or modified                |
%% |                                                              |
%% |  1. under the LaTeX Project Public License and/or            |
%% |  2. under the GNU Public License.                            |
%% |                                                              |
%% |  See the file doc/licenses/LICENSE for more details.         |
%% ----------------------------------------------------------------
%%
%% Version 0.1    03/28/2014    Initial presentation using this theme
%% Version 0.2    04/01/2014    First version to be published in the institute 
%%

%%%%%%%%%%%%%%%%%%%%%%%%%%%%%%%%%%%%%
%% Select input file encoding:
%%   utf8   - UTF-8, nowadays standard on most operating sytems
%%   latin1 - ISO-8859-1
\usepackage[utf8]{luainputenc}                 
%%%%%%%%%%%%%%%%%%%%%%%%%%%%%%%%%%%%%
%% Select language
%%
%\usepackage[ngerman]{babel} 				% Deutsch, neue Rechtschreibung
\usepackage[english]{babel} 				% English

\usetheme{gauss}

%% Do NOT change the following three lines, unless you know what you are doing
\usepackage[T1]{fontenc} 					% Font encoding
\usepackage{lmodern} 						% Select Linux Modern Fonts
\usepackage{graphicx} 						% needed to include graphics

%%%%%%%%%%%%%%%%%%%%%%%%%%%%%%%%%%%%%
%% import packages for content
%%
\usepackage{listings} 						% for lstlisting and \lstinline|..|
%% TikZ can be used to /program/ graphics.
\usepackage{tikz} 							% comment-out, if not needed 

%% Some TikZ-libraries and settings for the examples...
\usetikzlibrary{shadings} 					% GW: color gradients
\usetikzlibrary{calc,%
				positioning,%
				fit,%
				matrix,%
				shadows,%
				chains,%
				arrows,%
				shapes,%
				spy,%
				fadings}
\usepackage{pgfplots}
\usetikzlibrary{pgfplots.units,shapes.symbols,shapes.arrows}
%\usetikzlibrary{pgfplots.external}
%\tikzexternalize[prefix=tmp/]


%%%%%%%%%%%%%%%%%%%%%%%%%%%%%%%%%%%%%
%% configure title page and author information
%%-------------------------------
%% You can always provide a short version: \title[short]{long title}
%%   title        -- Title of the presentation
%%                   The title appears on the first page and may contain 
%% 					 a line break: \\ 
%%                   The short title appears in the footer line
%%   subtitle     -- Appears below the title
%%   titlegraphic -- Currently not supported
%%   author       -- Name of the author(s)
%%   email        -- E-Mail address of author (optional)
%%   institute    -- Name of the institution (e.g. chair)
%%   webaddress   -- Web address (default is www.rwth-aachen.de), 
%% 					 (see slide generated with \lastpage)
%%   date         -- Date of the presentation 
%% 					 (or use \date to insert the date of the PDF generation)
%%   subject      -- This is only for the PDF meta data
%%   keywords     -- This is only for the PDF meta data
%%   logo         -- Logo, don't change (given by coporate identity templates)

%\title[RWTH presentation template]{The E.ON ERC Latex \\ Presentation Template with Examples}
%\subtitle{Subtitle}
\usepackage{relsize}
\newcommand{\fast}{F{\smaller A}ST }

\title{Migration Framework}
\subtitle{Work Package 2}
\author{Simon Pickartz}
\email{spickartz@eonerc.rwth-aachen.de} 							% optionally
\institute[]{The \fast Project}
\instlogo{logos/fast}
\webaddress{www.eonerc.rwth-aachen.de} 						% overrides rwth-aachen.de 
\date{03/05/2015}
\subject{Migration Framework}          
\keywords{migration, HPC, mqtt}
% \instlogo{logos/acs}        % optionally


\begin{document}

%%
%% Every slide is written in a frame environment (\begin{frame}...\end{frame}).
%% You should provide a frame title after \begin{frame}.
%%

%%
%% Table of contents (automatically collects all \section{} and \subsection{} entries).
%% (run pdflatex multiple times to get all cross references correct)
%%
\begin{frame}{Agenda}
	\tableofcontents
\end{frame}

\section{Architecture}
\subsection{Overview}
\begin{frame}{Overview}
	\begin{itemize}
		\item Hier koennte z.\,B. ein Schaubild hin, wie man mit dem Migrations-Framwork interagiert.
		\item Eine zweite Folie könnte die Klassenstruktur etwas erklären und deutlich machen, was man benötigt um das Framwork in existierende andere Projekte zu integrieren
	\end{itemize}
\end{frame}

\begin{frame}[fragile]{YAML-Messages}
	\begin{columns}[T]
		\begin{column}{.40\textwidth}
			\begin{lstlisting}[basicstyle=\ttfamily, language=C]
			---
			task: migrate start
			vm-name: anthe1
			destination: pandora2
			parameter:
			  live-migration: false
			...
			\end{lstlisting}
		\end{column}
		\begin{column}{.40\textwidth}
			\begin{lstlisting}[basicstyle=\ttfamily, language=C]
			---
			result: migrate done
			list:
			  - vm-name: anthe1
			    status: success
			...
			\end{lstlisting}
		\end{column}
	\end{columns}
\end{frame}

\begin{frame}{Procedure}
	% define some styles
	\tikzstyle{format} = [draw, thin, fill=blue!20]
	\tikzstyle{medium} = [ellipse, draw, thin, fill=green!20, minimum height=2.5em]

	\begin{figure}
		\begin{tikzpicture}[node distance=3cm, auto,>=latex', thick]
			%\path[use as bounding box] (-1,0) rectangle (10,-2);
			\path[->] node[medium] (sched) {scheduler};
			\path[->] node[format, right of=sched] (comm) {communicator}
			(sched) edge node {MQTT} (comm);
			\path[->] node[format, right of=comm] (parser) {parser}
			(comm) edge node {YAML} (parser);
			\path[->] node[format, right of=parser] (th) {task handler}
			(parser) edge node {task} (th);
			\path[->] node[format, below of=th] (task) {task}
			(th) edge node {execute} (task);
			\path[->] node[format, below of=task] (hv) {hypervisor}
			(task) edge node {} (hv);
			\path[->] node[format, left of=task] (parser2) {parser}
			(task) edge node {result} (parser2);
			\path[->] node[format, left of=parser2] (comm2) {communicator}
			(parser2) edge node {YAML} (comm2);
			\path[->] node[medium, left of=comm2] (sched2) {scheduler};
			(comm2) edge node {MQTT} (sched2);
		\end{tikzpicture}
	\end{figure}
\end{frame}

\begin{frame}{Classes}
	\begin{itemize}
		\item Task\_handler is the main loop which gets messages from Communicator, calls parser and launches Tasks in Threads.
		\item Parser translates YAML messages to Task/Subtask objects and Result objects back to YAML.
		\item Communicator and Hypervisor are interfaces
			\begin{itemize}
				\item Implementations: MQTT\_communicator and Libvirt\_hypervisor
			\item By adding other implementations the framework can be expanded.
			\item Integration in other projects is also possible.
			\item e.g. using the Communicator interface to call migration
		\end{itemize}
		\item Tasks can contain more than one subtask (e.g. "start 5 VMs")
			\begin{itemize}
				\item each subtask is executed in a separate thread
			\end{itemize}
	\end{itemize}
\end{frame}


\subsection{Features}
\begin{frame}{Features}
%		\item Welche funktionalität wird bereits unterstüzt? Was fehlt noch?
%		\item Auf weiteren Folien könntest du details zu Besonderheiten erklären.
	\begin{columns}[T]
		\begin{column}{.40\textwidth}
			Done:
			\begin{itemize}
				\item All parts are ready (MQTT\_communicator, parser, Task\_handler, Task, Libvirt\_hypervisor)
				\item Task execution and communication are asynchronous.
				\item Start, stop and migrate tasks are currently available.
			\end{itemize}
		\end{column}
		\begin{column}{.40\textwidth}
		Missing:
		\begin{itemize}
			\item No details about errors are sent back.
			\item No possibility to ask status of migration framework
			\item No PCI attach/detach
			\item Other pre/post hooks (e.g. reduce memory)
		\end{itemize}
		\end{column}
	\end{columns}
\end{frame}

\section{Performance Results}
\pgfplotstableread[col sep=space]{data/libvirt/migrate_p3p4_anthe1fc20.dat}\migrationTime
\begin{frame}{Performance Results}
	\begin{itemize}
		\item Was kann man deiner Meinung nach alles Messen?
		\item Wichtig sind vergleichbare Ergebnisse.
			So solltest du dich z.\,B. für eine OS, Kernerl, Treiber, etc. Konfiguration entscheiden.
		\item Auf der nächsten Folie habe ich dir mal ein Beispiel angehängt, wie man Graphen mit tikz erzeugen kann.
	\end{itemize}
\end{frame}

\begin{frame}{Migration Time}
	\begin{center}
		\begin{tikzpicture}
%			\path[use as bounding box] (0em, -2em) rectangle (15em, 15em);
			\tiny
			\begin{semilogxaxis}[
					height=\textheight,
					width=.9\textwidth,
					axis x line=bottom,
					axis y line=left,
					log basis x=2,
					ymajorgrids=true,
					legend style={font=\tiny},
					legend cell align=left,
%
					legend pos=north west,
					ymin=0, ymax=35,
					xmin=134217728, xmax=51539607552,%34359738368,
					xlabel={Memory in Byte}, 
					ylabel={Migration time in s},
					xtick={134217728  , 536870912  , 2147483648 , 8589934592 ,  34359738368},
					xticklabels={,512\,Mi,2\,Gi,8\,Gi,32\,Gi},
				]
				\addplot [
					thick, 
					mark=*,
					color=light,
					mark color=light,
				] table[x=mem_assigned,y=cold]{\migrationTime};
				\addlegendentry{Migration only}
				\addplot [
					thick, 
					mark=triangle*,
					color=dark,
					mark color=dark,
				] table[x=mem_assigned,y=cold_hotplug]{\migrationTime};
				\addlegendentry{With hotplug}
			\end{semilogxaxis}
		\end{tikzpicture}
	\end{center}
\end{frame}

\section{Outlook}
\begin{frame}{Outlook}
	\begin{itemize}
		\item Basics are working
		\item PCI attach/detach almost done
		\item Currently working on verbose error reporting
%		\item Du kannst es auch ``Conclusion'' nennen
%		\item Hier kannst du noch einmal kurz zusammen fassen was bisher erreicht wurde und was noch fehlt.
	\end{itemize}
\end{frame}

%%
%% the final slide (contact information) can be generated with the following
%% command
%%
\lastpage
\appendix
\section{Example slides}

\begin{frame}{Title of the frame that may also go into the second line}
	\begin{itemize}
		\item Itemize list 
		\item Itemize list
			\begin{itemize}
				\item Second layer 
					\begin{itemize}
						\item Third layer 
					\end{itemize}
			\end{itemize}
		\item Itemize List 
		\item Itemize List 
			\begin{itemize}
				\item Second layer
			\end{itemize}
	\end{itemize}
\end{frame}

\begin{frame}{Title of a frame having paragraphs}
	\begin{itemize}
		\item Lorem ipsum dolor sit amet, consectetur adipisicing elit, sed do eiusmod tempor incididunt ut labore et dolore magna aliqua.
		\item Lorem ipsum dolor sit amet, consectetur adipisicing elit, sed do eiusmod tempor incididunt ut labore et dolore magna aliqua.
		\item Lorem ipsum dolor sit amet, consectetur adipisicing elit, sed do eiusmod tempor incididunt ut labore et dolore magna aliqua.
	\end{itemize}
\end{frame}

\begin{frame}{Frame with boxes}
	\begin{block}{Block 1}
		Content of the first block
	\end{block}
	\begin{block}{Another block}
		Lorem ipsum dolor sit amet, consectetur adipisicing elit, sed do eiusmod tempor incididunt ut labore et dolore magna aliqua.
	\end{block}
\end{frame}

\section{More Examples}
%% 
%% If a \subsection{} is available, that one is used as first line of the frame title.
%% 
\subsection{A Subsection}

\begin{frame}{How to set columns}
  %%
  %% The columns environment is provided by beamer, 
  %% see "texdoc beamer", Sec. 12.7 'Splitting a Frame into Multiple Columns'
  %%
  %% parameter: c - center columns vertically
  %%            t - align columns on the baseline of the first line
  %%                don't use, if a column contains (only) a graphic! 
  %%            T - align columns on the top of the first line (ok with graphics)
  %%            b - align columns on the bottom line
	\begin{columns}[T]
	%% Each column must be given a with.
	%% Should be given relative to \textwidth:
		\begin{column}{.40\textwidth}
			\begin{itemize}
				\item Item 1
				\item Item 2
				\item Item 3
				\item Lorem ipsum dolor sit amet, consectetur adipisicing elit, sed do eiusmod tempor incididunt ut labore et dolore magna aliqua. 
			\end{itemize}
		\end{column}
		\begin{column}{.40\textwidth}
	  %% here, \textwidth is the with of the current column
			\includegraphics[width=.9\textwidth]{pictures/Tux}
		\end{column}
	\end{columns}
\end{frame}


\begin{frame}{How to set columns (2)}
	Lorem ipsum dolor sit amet, consectetur adipisicing elit, sed do eiusmod tempor incididunt ut labore et dolore magna aliqua. 
	\begin{columns}[t]
		\begin{column}{.35\textwidth}
			\begin{itemize}
				\item cat
				\item dog
				\item mouse
				\item elephant
			\end{itemize}
		\end{column}
		\begin{column}{.55\textwidth}
			\begin{itemize}
				\item Lorem ipsum dolor sit amet, consectetur adipisicing elit, sed do eiusmod tempor incididunt ut labore et dolore magna aliqua. 
				\item Lorem ipsum dolor sit amet, consectetur adipisicing elit, sed do eiusmod tempor incididunt ut labore et dolore magna aliqua. 
			\end{itemize}
		\end{column}
	\end{columns}

	Lorem ipsum dolor sit amet, consectetur adipisicing elit, sed do eiusmod tempor incididunt ut labore et dolore magna aliqua. 
\end{frame}

\subsection{another subsection}

\begin{frame}
	This frame does not contain a (dedicated) frame title.
\end{frame}

\section{Examples with uncovering}

\begin{frame}{Piecewise uncovering using pause}
	Paragraphs and items can be uncovered easily with \texttt{\textbackslash pause}.
	\pause
	\begin{itemize}
		\item Lorem ipsum dolor sit amet, consectetur adipisicing elit, sed do eiusmod tempor incididunt ut labore et dolore magna aliqua. 
			\pause
		\item Lorem ipsum dolor sit amet, consectetur adipisicing elit, sed do eiusmod tempor incididunt ut labore et dolore magna aliqua. 
			\pause
		\item Lorem ipsum dolor sit amet, consectetur adipisicing elit, sed do eiusmod tempor incididunt ut labore et dolore magna aliqua. 
			\pause
		\item Lorem ipsum dolor sit amet, consectetur adipisicing elit, sed do eiusmod tempor incididunt ut labore et dolore magna aliqua. 
	\end{itemize}
\end{frame}

\begin{frame}{Fine grained control}
	\begin{itemize}
		\item<1-> First line
		\item<2>  Second line (appears after first line, disappears again)
		\item<3-4> Third line (appears before firth one)
		\item<4-> Fourth line
		\item<5-> Fifth line
	\end{itemize}
	\onslide<6->{
		Standard behavior with \texttt{\textbackslash onslide} is to keep its place, even if not displayed.
	}

	\only<7>{
		With \texttt{\textbackslash only}, the element does not occupy place. This may lead to shifting other elements.
	}
\end{frame}

\section{Quellcode}

%% 
%% This requires the package listings
%%
%% Slides containing lstlisting environments, \lstinline|..| or \verb|..|
%% the option "fragile" must be provided!
%%
\begin{frame}[fragile]{Example-Code: Hallo Welt!}
	\begin{lstlisting}[basicstyle=\ttfamily, language=C]
	#include <stdio.h>

	void main (void) {
	    printf("Hallo Welt!\n");
	}
	\end{lstlisting}
\end{frame}




\section{TikZ example}

%% 
%% This requires the package tikz
%%
%% The animation uses the same overlay specification<2-3> as the items or \onslide
%%
\begin{frame}{animated graphics}

  % define some styles
	\tikzstyle{format} = [draw, thin, fill=blue!20]
	\tikzstyle{medium} = [ellipse, draw, thin, fill=green!20, minimum height=2.5em]

	\begin{figure}
		\begin{tikzpicture}[node distance=3cm, auto,>=latex', thick]
	  % We need to set a bounding box first. Otherwise the diagram
	  % will change position for each frame.
			\path[use as bounding box] (-1,0) rectangle (10,-2);
			\path[->]<1-> node[format] (tex) {.tex file};
			\path[->]<2-> node[format, right of=tex] (dvi) {.dvi file}
			(tex) edge node {\TeX} (dvi);
			\path[->]<3-> node[format, right of=dvi] (ps) {.ps file}
			node[medium, below of=dvi] (screen) {screen}
			(dvi) edge node {dvips} (ps)
			edge node[swap] {xdvi} (screen);
			\path[->]<4-> node[format, right of=ps] (pdf) {.pdf file}
			node[medium, below of=ps] (print) {printer}
			(ps) edge node {ps2pdf} (pdf)
			edge node[swap] {gs} (screen)
			edge (print);
			\path[->]<5-> (pdf) edge (screen)
			edge (print);
			\path[->, draw]<6-> (tex) -- +(0,1) -| node[near start] {pdf\TeX} (pdf);
		\end{tikzpicture}
	\end{figure}

\end{frame}


\end{document}
